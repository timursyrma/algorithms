

\section{Кучи в алгоритмах}\label{sec:--}

\subsection{Введение в двоичные кучи}\label{subsec:---2}

\begin{definition}
    Двоичная куча - двоичное сбалансированное дерево, которое хранится в массиве и строится по 2 правилам:
    \begin{itemize}
        \item Для любого $i < \lfloor \large{\frac{heapSize}{2}} \rfloor$ следует  $\, heap[i] \le heap[2\cdot i+1]$ и $heap[i] \le heap[2\cdot i+2]$
        \item Нижний слой кучи заполняется слева направо, но если он заполнен,
        то верщина добавляется как ребёнок самой левой вершины нижнего слоя.
    \end{itemize}
\end{definition}

\subsection{Операции на двоичной куче}\label{subsec:---3}

\begin{operation}
    Операция добавления элемента в кучу происходит в несколько этапов:
    \begin{enumerate}
        \item Мы вставляем элемент в кучу по второму вышеуказаному правилу.
        \item Затем выполняем операцию \hyperlink{su}{sift up} для этого элемента.
    \end{enumerate}
    Псевдокод:
    \begin{algorithmic}[0]
        \Function{add}{value}
            \State heap[heapSize] $\gets$ value;
            \State ++heapSize;
            \State siftUp(heapSize - 1);
        \EndFunction
    \end{algorithmic}
\end{operation}

\begin{operation}
    Операция удаления минимального элемента из кучи происходит в несколько этапов:
    \begin{enumerate}
        \item Мы меняем местами первый и последний элемент в массиве куче.
        \item Далее уменьшаем размер кучи, тем самым удаляем последний элемент.
        \item Затем выполняем операцию \hyperlink{sd}{sift down}
        для этого первого элемента в куче для восстановления свойств кучи.
    \end{enumerate}
    Псевдокод:
    \begin{algorithmic}[0]
        \Function{removeMin}{\,}
            \State swap(heap[0], heap[heapSize - 1]);
            \State --heapSize;
            \State siftDown(0);
        \EndFunction
    \end{algorithmic}
\end{operation}

\begin{operation}
    Операция удаления произвольного элемента из кучи происходит в несколько этапов:
    \begin{enumerate}
        \item Мы меняем местами i-ый и последний элемент в массиве куче.
        \item Далее уменьшаем размер кучи, тем самым удаляем последний элемент.
        \item Затем выполняем операции \hyperlink{sd}{sift down} и \hyperlink{su}{sift up}
        для i-ого в куче для восстановления свойств кучи.
    \end{enumerate}
    Псевдокод:
    \begin{algorithmic}[0]
        \Function{removeByIndex}{index}
            \State swap(heap[index], heap[heapSize - 1]);
            \State --heapSize;
            \State siftDown(index);
            \State siftUp(index);
        \EndFunction
    \end{algorithmic}
\end{operation}

\begin{operation}
    Построение кучи на произвольном массиве за $O(n)$ происходит в несколько этапов:
    \begin{enumerate}
        \item Выполняем цикл от $\large{\frac{heapSize}{2}} $ до нуля.
        \item На каждой итерации цикла выполянем операцию \hyperlink{sd}{sift down}.
    \end{enumerate}
    Псевдокод:
    \begin{algorithmic}[0]
        \Function{buildHeap}{\,}
            \For{$i \gets \large{\frac{heapSize}{2}},\dots,0$}
                \State siftDown(i);
            \EndFor
        \EndFunction
    \end{algorithmic}
\end{operation}

\newpage
\subsection{Восстановление свойств двоичных кучах}\label{subsec:---4}

\begin{operation}
    \hypertarget{su}{}
    Операция sift up в куче нужна,
    чтобы восстановить указанные в определении свойства кучи для i-ого элемента и всех его предков.
    \begin{enumerate}
        \item Мы берём i-ый элемент и проверяем, что его родитель - $\frac{i - 1}{2}$ элемент меньше его.
        Если это так, то алгоритм прекращается. Либо, если $i = 0$, то в таком случаем алгоритм также прекращается,
        иначе переходим к пункту 2.
        \item Если условие не выполнилось, то меняем местами значения $heap[i]$ и $heap[\frac{i - 1}{2}]$.
        Затем повторяем 1 пункт для $\frac{i - 1}{2}$ элемента.
    \end{enumerate}
    Псевдокод:
    \begin{algorithmic}[0]
        \Function{siftUp}{childIndex}
            \If{childIndex == 0}
                \State return;
            \EndIf\\
            \State parentIndex $\gets$ \large{$\frac{childIndex - 1}{2}$};
            \If{heap[childIndex] < heap[parentIndex]}
                \State swap(heap[childIndex], heap[parentIndex]);
                \State siftUp(patentIndex);
            \EndIf

        \EndFunction
    \end{algorithmic}
\end{operation}

\begin{operation}
    \hypertarget{sd}{}
    Операция sift down в куче нужна,
    чтобы восстановить указанные в определении свойства кучи для i-ого элемента и всех его детей.
    \begin{enumerate}
        \item Мы берём i-ый элемент и проверяем, что его левый ребёнок - $2\cdot i + 1$ - существует.
        \item Если он не существует, то завершаем алгоритм.
        \item Если он существует, то выбираем наибольший из существующих двух детей и меняем местами с $heap[i]$.
        \item Повторяем пункт 1 для изменённого ребёнка.
    \end{enumerate}

    \begin{algorithmic}[0]
        \Function{siftDown}{parentIndex}
        \State leftChildIndex $\gets$ parentIndex$\cdot 2 + 1$;
        \If{lefChildIndex $\ge$ heapSize}
            \State return;
        \EndIf\\
        \State rightChildIndex $\gets$ parentIndex$\cdot 2 + 2$;
        \State swapChildIndex $\gets$ leftChildIndex;
        \If{rightChildIndex < heapSize AND heap[leftChildIndex] < heap[rightChildIndex]}
            \State swapChildIndex $\gets$ rightChildIndex;
        \EndIf\\

        \If{heap[swapChildIndex] < heap[parentIndex]}
            \State swap(heap[swapChildIndex], heap[parentIndex]);
            \State siftDown(swapChildIndex);
        \EndIf

        \EndFunction
    \end{algorithmic}
\end{operation}


