\section{Оценка сложности алгоритмов}

\subsection{Введение в асимптотику}

\begin{definition}
	$f = O(g)$, если $\exists N > 0, c > 0 : \forall n \ge N, f(n) \le c \cdot g(n)$.
\end{definition}
\begin{definition}
	$f = \Omega(g)$, если $\exists N > 0, c > 0 : \forall n \ge N, f(n) \ge c \cdot g(n)$.
\end{definition}
\begin{definition}
	$f = \Theta(g)$, если $\exists N > 0, c_1 > 0, c_2 > 0: \forall n \ge N, c_1 \le f(n) \le c_2 \cdot g(n)$.
\end{definition}

Вместо знака равенства можно употреблять знаки принадлежности и включения $\in$, понимая $O, \Omega, \Theta$ как обозначение для множеств функций.
\subsection{Второй раздел}


\begin{theorem}[Великая теорема Ферма]
    Уравнение $x^n + y^n = z^n$ не имеет нетривиальных целочисленных решений при $n \in \N$, $n > 2$.
\end{theorem}

\begin{proof}
    Тривиально.
\end{proof}

